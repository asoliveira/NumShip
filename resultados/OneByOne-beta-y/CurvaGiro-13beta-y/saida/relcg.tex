\chapter{Relat�rio da Curva Giro Navio B MarAd }

	\section{Curva de Giro}
	\subsection{Dados}
	\begin{center}
	\begin{tabular}{ll}\ 
		Di�metro t�tico  = &     981.357311\\ 
		Avan�o  = &   898.555230\\ 
		Transfer�ncia  = &   400.920571\\ 
		Raio da curva de equil�brio = &   440.471632\\ 
		\end{tabular}
	\end{center}\begin{figure}[H]
		\includegraphics[scale=0.7]{./figuras/Curva_de_Giro/MARADPosOri.eps}
	\caption{\textit{Curva de Giro}}
	\end{figure}
	\subsection{Velocidades}

	\begin{figure}[H]
		\includegraphics[scale=0.7]{./figuras/Curva_de_Giro/MARADVelo.eps}
	\caption{\textit{Curva de Giro}}
	\end{figure}

	\subsection{Acelera��es}
	\begin{figure}[H]
		\includegraphics[scale=0.7]{./figuras/Curva_de_Giro/MARADAcel.eps}
	\caption{\textit{Curva de Giro}}
	\end{figure}

	\subsection{For�as e Momentos}
	\begin{figure}[H]
		\includegraphics[scale=0.7]{./figuras/Curva_de_Giro/MARADForMom.eps}
	\caption{\textit{Curva de Giro}. As for�as e os mometos est�o multiplicados por 1.0}
	\end{figure}

	\subsection{Eta}
	\begin{figure}[H]
		\includegraphics[scale=0.7]{./figuras/Curva_de_Giro/MARADpltetat.eps}
	\caption{\textit{Curva de Giro }}
	\end{figure}

	